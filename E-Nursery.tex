\documentclass[12pt]{report}
\usepackage{hyperref}
\usepackage{blindtext}
\usepackage{indentfirst}
\usepackage{graphicx}
\begin{document}
    \tableofcontents
    \chapter{Introduction}
    \section{Introduction to the System}
    \paragraph{
    This project is aimed at developing a Web application that depicts online Shopping of plants, seeds, fertilizers, flowers etc. products.Using this software, companies can improve the efficiency of their services. Online Shopping is one of the applications to improve the marketing and sale of the company’s products. This web application involves all the basic features of the online shopping.
    }
    \section{Problem Definition}
    \paragraph
    {
    Managing your Online Nursery Store System may seem tricky, but this is part of Customer Service System (application support direct contact with customer)
    }
    \section{Aim}
    \paragraph
    {
    To manage online shopping of plants, flowers, seeds and fertilizers
    }
    \section{Objective}
    \paragraph
    {
    This software helps customer to find different nursery products according category, occasion, price and subcategory. It is designed such a way that one can view all the updates of the products from any place through online. The software will help in easy maintaining and updating products in the website for the administrator. Also quick and easy comparison of different products for the customers.
    }
    \section{Goal}
    \begin{itemize}
        \item The project is basically targeted at those people who would like online shopping and have an   Internet access.
        
        \item Finally buyers curious in comparing the prices for various products for according our budget. 
        
        \item To make a database that is consistent, reliable and secure.
        
        \item To provide correct, complete, ongoing information.
        
        \item To develop a well-organized information storage system.
        
        \item To make good documentation so as to facilitate possible future enhancements.
    \end{itemize}
    \section{Need of System}
    \paragraph
    {
    There is always a need of a system that will perform to purchasing products online according to occasion. This system will reduce the manual operation required to maintain all the records of booking information. And also generates the various reports for analysis. Main concept of the project is to enter transaction reports and to maintain customer records. Hence this software can be used in any nursery store to maintain their record easily.
    }
    \chapter{Hardware and Software requirement}
        \section{Introduction}
        \paragraph
        {
        In this chapter we mentioned the software and hardware requirements, which are necessary for successfully running this system. The major element in building systems is selecting compatible hardware and software. The system analyst has to determine what software package is best for the “E Nursery System” and, where software is not an issue, the kind of hardware and peripherals needed for the final conversion.
        }
        \section{System environment}
        \paragraph
        {
        After analysis, some resources are required to convert the abstract system into the real one. All the resources, which accomplish a robust The hardware and software selection begins with requirement analysis, followed by a request for proposal and vendor evaluation.
        }
        \paragraph
        {
        Software and real system are identified. According to the provided functional specification all the technologies and its capacities are identified. Basic functions and procedures and methodologies are prepared to implement. Some of the Basic requirements such as hardware and software are described as follows: -
        }
        \paragraph{\textbf{Hardware and Software Specification}}
        \paragraph{\underline{\textbf{Software Requirements:}}}
        \section{Software requirement}
        \begin{itemize}
            \item Framework: Django
            \item IDE : Pycharm/Atom
            \item Client Side Technologies: HTML, CSS, JavaScript , Bootstrap
            \item Server Side Technologies: Python
            \item Data Base Server: Sqlite
            \item Operating System: Microsoft Windows/Linux 
        \end{itemize}
        
        
        \section{Hardware requirements}
       \begin{itemize}
           \item Processor: Pentium-III (or) Higher
           \item Ram: 64MB (or) Higher
           \item Hard disk: 80GB (or) Higher
       \end{itemize}
    \chapter{System Analysis}
        \section{Purpose}
        \paragraph
        {
        The main purpose of this web project is to manage the online shopping of nursery products. It helps to customer to order products from anywhere at anytime. Also make payment on delivery for it. It helps to people to order desired flower at their prefer time.
        }
        \section{Project Scope}
        \paragraph
        {
        The project has a wide scope, as it is not intended to a particular organization. This project is going to develop generic software, which can be applied by any businesses organization. More over it provides facility to its customer. Also the software is going to provide a huge amount of summary data.
        }
        \section{Proposed System}
        \paragraph
        {
        The e nursery system is available in the market that can serve customers to order nursery products online.  The development of the new system contains the following activities, which try to automate the entire process keeping in view of the database integration approach.
        }
        \begin{enumerate}
            \item User friendliness is provided in the application with various controls.
            
            \item The system makes the overall project management much easier and flexible
            
            \item There is no risk of data mismanagement at any level while the project development is under process.
            
            \item It provides high level of security with different level of authentication.
        \end{enumerate}
        \section{System Description}
        \paragraph{The key features required in the system are as follows:}
        \begin{itemize}
            \item \textbf{Login:} This module has a drop down list box from where we have to select \textbf{ADMIN or USER}. The \textbf{ADMIN} has all the rights in the software including updating the status of his site. The other fields in login are username and password. If the username and password are correct then it is directed to next page.
            \item \textbf{New user:} This module is for the users who do not have their account. Here user is allowed to create an account to login. The account creation is done by filling the registration form with user details such as name, phone, email etc.
            \item \textbf{Product:} This module has information regarding the product such as its name, category, subcategory, image, price information; its features etc.The \textbf{ADMIN} has the authority to Add, Delete, and Update etc. The \textbf{USER} can only view the product available in the stock etc.
            \item \textbf{Search:} This module helps the customer to ease his search based on his budget or interest. The search can be done on different categories and subcategories like category , subcategory , name, price etc
        \end{itemize}
    \chapter{Implementation issues}
        \section{Python}
        \paragraph
        {
        Python is a widely used general-purpose, high level programming language. It was initially designed by Guido van Rossum in 1991 and developed by Python Software Foundation. It was mainly developed for emphasis on code readability, and its syntax allows programmers to express concepts in fewer lines of code.
        }
        \paragraph
        {
        Python is a programming language that lets you work quickly and integrate systems more efficiently. Python is dynamically typed and garbage-collected. It supports multiple programming paradigms, including procedural, object-oriented, and functional programming. Python is often described as a "batteries included" language due to its comprehensive standard library.
        }
        \section{HTML}
        \paragraph
        {
        HTML (Hypertext Markup Language) is the set of markup symbols or codes inserted in a file intended for display on a World Wide Web browser page. The markup tells the Web browser how to display a Web page's words and images for the user. Each individual markup code is referred to as an element (but many people also refer to it as a tag). Some elements come in pairs that indicate when some display effect is to begin and when it is to end.
        }
        \section{Cascading style sheet(CSS)}
        \paragraph
        {
        Cascading Style Sheets (CSS) are a collection of rules we use to define and modify web pages.  CSS are similar to styles in Word.  CSS allow Web designers to have much more control over their pages look and layout.  For instance, you could create a style that defines the body text to be Verdana, 10 point. Later on, you may easily change the body text to Times New Roman, 12 point by just changing the rule in the CSS.  Instead of having to change the font on each page of your website, all you need to do is redefine the style on the style sheet, and it will instantly change on all of the pages that the style sheet has been applied to. With HTML styles, the font change would be applied to each instance of that font and have to be changed in each spot.  
        }
        \paragraph
        {
        CSS can control the placement of text and objects on your pages as well as the look of those objects.
        }
        \paragraph
        {
        HTML information creates the objects (or gives objects meaning), but styles describe how the objects should appear. The HTML gives your page structure, while the CSS creates the “presentation”.  An external CSS is really just a text file with a .css extension.  These files can be created with Dreamweaver, a CSS editor, or even Notepad.
        }
        \paragraph
        {
        The best practice is to design your web page on paper first so you know where you will want to use styles on your page. Then you can create the styles and apply them to your page.
        }
        \section{Javascript}
        \paragraph
        {
        JavaScript is a programming language commonly used in web development. It was originally developed by Netscape as a means to add dynamic and interactive elements to websites. While JavaScript is influenced byJava, the syntax is more similar to C and is based on ECMAScript, a scripting language developed by Sun Microsystems. 
        }
        \paragraph
        {
        JavaScript is a client-side scripting language, which means the source code is processed by the client's web browser rather than on the web server. This means JavaScript functions can run after a webpage has loaded without COMMUNICATING with the server. For example, a JavaScript function may check a web form before it is submitted to make sure all the required fields have been filled out. The JavaScript code can produce an error message before any information is actually transmitted to the server. 
        }
        \paragraph
        {
        Like server-side scripting languages, such as PHP and ASP, JavaScript code can be inserted anywhere within the HTML of a webpage. However, only the output of server-side code is displayed in the HTML, while JavaScript code remains fully visible in the source of the webpage. It can also be referenced in a separate .JS file, which may also be viewed in a browser.
        }
        \section{Django}
        \paragraph
        {
        Django is a web application framework written in Python programming language. It is based on MVT (Model View Template) design pattern. The Django is very demanding due to its rapid development feature. It takes less time to build application after collecting client requirement.
        }
        \paragraph
        {
        This framework uses a famous tag line: The web framework for perfectionists with deadlines.
        }
        \section{Sqlite}
        \paragraph
        {
        SQLite is an in-process library that implements a self-contained, serverless, zero-configuration, transactional SQL database engine. The code for SQLite is in the public domain and is thus free for use for any purpose, commercial or private. SQLite is the most widely deployed database in the world with more applications than we can count, including several high-profile projects.
        }
    \chapter{System Design}
        \section{use case diagram}
        \paragraph
        {
        Use case diagram consists of use cases and actors and shows the interaction between them. The key points are: }
        \begin{itemize}
            \item The main purpose is to show the interaction between the use cases and the actor.
            \item To represent the system requirement from user’s perspective. 
            \item The use cases are the functions that are to be performed in the module. 
        \end{itemize}
        \begin{center}
            \includegraphics{admin system use case.jpg}
            \includegraphics{user system use case.jpg}
        \end{center}

        \section{Sequence Diagram}
        \begin{center}
            \includegraphics{sequence diagram.jpg}
            \textbf{\underline{Sequence Diagram for Administrator}}
            \paragraph{}
            \includegraphics{sequence diagram for user.jpg}
            \textbf{\underline{Sequence Diagram for Users}}
        \end{center}
        
        \section{Data Flow Diagram}
        \paragraph
        {
        A Data Flow Diagram (DFD) is a graphical representation of the "flow" of data through an Information System. A data flow diagram can also be used for the visualization of Data Processing. It is common practice for a designer to draw a context-level DFD first which shows the interaction between the system and outside entities. This context-level DFD is then "exploded" to show more detail of the system being modeled.
        }
        \paragraph
        {
        A DFD represents flow of data through a system. Data flow diagrams are commonly used during problem analysis. It views a system as a function that transforms the input into desired output. A DFD shows movement of data through the different transformations or processes in the system.
        }
        \paragraph
        {
         Dataflow diagrams can be used to provide the end user with a physical idea of where the data they input ultimately has an effect upon the structure of the whole system from order to dispatch to restock how any system is developed can be determined through a dataflow diagram. The appropriate register saved in database and maintained by appropriate authorities.
        }
        
        
        \begin{center}
        \includegraphics{DFD Notations.jpg}    
        \end{center}
        
        \pagebreak
        \begin{center}
            \textbf{\underline{\large{DFD for E-Nursery}}}
        \end{center}
        \begin{center}
            \includegraphics{dfd 0.jpg}  
            \textbf{\underline{DFD level 0}}
        \end{center}
        
        \pagebreak
        \begin{center}
            \includegraphics{DFD 1.jpg}
            \textbf{\underline{DFD level 1}}
        \end{center}
        \pagebreak
        \section{ER-Diagram}
        \paragraph
        {
        An entity-relationship (ER) diagram is a specialized graphic that illustrates the interrelationships between entities in a database. ER diagrams often use symbols to represent three different types of information. Boxes are commonly used to represent entities. Diamonds are normally used to represent relationships and ovals are used to represent attributes
        }
        \paragraph
        {
        An \textbf{entity-relationship model} (ERM) in software engineering is an abstract and conceptual representation of data. Entity-relationship modeling is a relational schema database modeling method, used to produce a type of conceptual schema or semantic data model of a system, often a relational database, and its requirements in a top-down fashion.
        }
        
        \paragraph
        {
        \textbf{\underline{Symbols used in this E-R Diagram:}}
        }
        
        \paragraph
        {
        \textbf{Entity: }Entity is a “thing” in the real world with an independent existence. An entity may be an object with a physical existence such as person, car or employee. Entity symbol is as follows
        }
        \begin{center}
        \includegraphics{rectangle.jpg}    
        \end{center}
        
        \paragraph
        {
        \textbf{Attribute: }Attribute is a particular property that describes the entity. Attribute symbol is
        }
        \begin{center}
        \includegraphics{oval.jpg}    
        \end{center}
        
        \paragraph
        {
        \textbf{Relationship: }Relationship will be several implicit relationships among various entity types whenever an attribute of one entity refers to another entity type some relationship exits. Relationship symbol is:
        }
        \begin{center}
            \includegraphics{diamond.jpg}
        \end{center}
        
        \paragraph
        {
        \textbf{Key attributes: }An entity type usually has an attribute whose values are distinct for each individual entity in the collection. Such an attribute is called key attribute. 
        }
       
        
        \begin{center}
            \pagebreak
            \textbf{\underline{ERD for E-Nursery}}    
            \includegraphics{ERD.jpg}
        \end{center}
    \chapter{User  Screens}
   
    \large \textbf{\underline{Home Page}}
    \begin{center}
        \includegraphics{home page.jpg}
    \end{center}
    
    \pagebreak
    \large \textbf{\underline{VIEW ALL PRODUCTS PAGE}}
    \begin{center}
        \includegraphics{view all product page.jpg}
    \end{center}
    
    \pagebreak
    \large \textbf{\underline{SignUp Page}}
    \begin{center}
        \includegraphics{sign up.jpg}
    \end{center}
    
    \pagebreak
    \large \textbf{\underline{Sign In Page}}
    \begin{center}
        \includegraphics{sign in.jpg}
    \end{center}
    
    \pagebreak
    \large \textbf{\underline{Profile Page}}
    \begin{center}
        \includegraphics{profile page.jpg}
    \end{center}
    
    \pagebreak
    \large \textbf{\underline{Change Password Page}}
    \begin{center}
        \includegraphics{change password.jpg}
    \end{center}
    
    \pagebreak
    \large \textbf{\underline{View All Products Page – User}}
    \begin{center}
        \includegraphics{view all product page.jpg}
    \end{center}
    
    \pagebreak
    \large \textbf{\underline{Confirm Booking Page}}
    \begin{center}
        \includegraphics{Confirm Booking Page.jpg}
    \end{center}
    
    \pagebreak
    \large \textbf{\underline{View My Orders Page - User}}
    \begin{center}
        \includegraphics{view all product page.jpg}
    \end{center}
    
    \pagebreak
    \large \textbf{\underline{Cart Page}}
    \begin{center}
        \includegraphics{cart page.jpg}
    \end{center}
    
    \pagebreak
    \large \textbf{\underline{Admin Login Page}}
    \begin{center}
        \includegraphics{Admin Login Page.jpg}
    \end{center}
    
    \pagebreak
    \large \textbf{\underline{Admin Dashboard Page}}
    \begin{center}
        \includegraphics{Admin Dashboard Page.jpg}
    \end{center}
    
    \pagebreak
    \large \textbf{\underline{View All Users Page}}
    \begin{center}
        \includegraphics{view all product page.jpg}
    \end{center}
    
    \pagebreak
    \large \textbf{\underline{Add New Category}}
    \begin{center}
        \includegraphics{Add New Category.jpg}
    \end{center}
    
    \pagebreak
    \large \textbf{\underline{Manage Product Category}}
    \begin{center}
        \includegraphics{Manage Product Category.jpg}
    \end{center}
    
    \pagebreak
    \large \textbf{\underline{Add Product Page}}
    \begin{center}
        \includegraphics{Add Product Page.jpg}
    \end{center}
    
    \pagebreak
    \large \textbf{\underline{Manage Products Page}}
    \begin{center}
        \includegraphics{Manage Products Page.jpg}
    \end{center}
    
    \pagebreak
    \large \textbf{\underline{View All Orders}}
    \begin{center}
        \includegraphics{View All Orders.jpg}
    \end{center}
    
    \chapter{Coding}
    \textbf{\underline{\large Home page coding}}
    \begin{center}
        \includegraphics{Home Page 1.jpg}
        \includegraphics{Home Page 2.jpg}
    \end{center}
    
    \pagebreak
    \textbf{\underline{\large USER REGISTRATION PAGE CODING}}
    \begin{center}
        \includegraphics{USER REGISTRATION PAGE CODING 1.jpg}
        \includegraphics{USER REGISTRATION PAGE CODING 2.jpg}
        \includegraphics{USER REGISTRATION PAGE CODING 3.jpg}
    \end{center}
    
    \pagebreak
    \textbf{\underline{\large LOGIN CODING}}
    \begin{center}
        \includegraphics{LOGIN CODING 1.jpg}
        \includegraphics{LOGIN CODING 2.jpg}
    \end{center}
    
    \pagebreak
    \textbf{\underline{\large LOGIN CODING}}
    \begin{center}
        \includegraphics{LOGIN CODING 1.jpg}
        \includegraphics{LOGIN CODING 2.jpg}
    \end{center}
    
    \pagebreak
    \textbf{\underline{\large VIEW CATEGORYWISE PRODUCTS PAGE CODING}}
    \begin{center}
        \includegraphics{VIEW CATEGORYWISE PRODUCTS PAGE CODING 1.jpg}
        \includegraphics{VIEW CATEGORYWISE PRODUCTS PAGE CODING 2.jpg}
        \includegraphics{VIEW CATEGORYWISE PRODUCTS PAGE CODING 3.jpg}
        \includegraphics{VIEW CATEGORYWISE PRODUCTS PAGE CODING 4.jpg}
        \includegraphics{VIEW CATEGORYWISE PRODUCTS PAGE CODING 5.jpg}
    \end{center}
    
    \pagebreak
    \textbf{\underline{\large CART PAGE CODING}}
    \begin{center}
        \includegraphics{CART page 1.jpg}
        \includegraphics{CART page 2.jpg}
        \includegraphics{CART page 3.jpg}
    \end{center}
    
     \pagebreak
    \textbf{\underline{\large VIEW MY ORDERS PAGE CODING}}
    \begin{center}
        \includegraphics{VIEW MY ORDERS PAGE CODING 1.jpg}
        \includegraphics{VIEW MY ORDERS PAGE CODING 2.jpg}
        \includegraphics{VIEW MY ORDERS PAGE CODING 3.jpg}
    \end{center}
    
    \pagebreak
    \textbf{\underline{\large ADMIN HOME PAGE CODING}}
    \begin{center}
        \includegraphics{Admin Home page 1.jpg}
        \includegraphics{Admin Home page 2.jpg}
        \includegraphics{Admin Home page 3.jpg}
    \end{center}
    
    \pagebreak
    \textbf{\underline{\large ADD PRODUCT PAGE CODING}}
    \begin{center}
        \includegraphics{Admin Home page 1.jpg}
        \includegraphics{Admin Home page 2.jpg}
        \includegraphics{Admin Home page 3.jpg}
    \end{center}
    
    \pagebreak
    \textbf{\underline{\large ADD PRODUCT PAGE CODING}}
    \begin{center}
        \includegraphics{ADD PRODUCT PAGE CODING 1.jpg}
        \includegraphics{ADD PRODUCT PAGE CODING 2.jpg}
    \end{center}
    
    \pagebreak
    \textbf{\underline{\large Manage Product Page Coding}}
    \begin{center}
        \includegraphics{Manage Product Page Coding 1.jpg}
        \includegraphics{Manage Product Page Coding 2.jpg}
    \end{center}
    
     \pagebreak
    \textbf{\underline{\large View All Users Page Coding}}
    \begin{center}
        \includegraphics{View All Users Page Coding 1.jpg}
        \includegraphics{View All Users Page Coding 2.jpg}
    \end{center}
    \chapter{Conclusion}
    \paragraph
    {
    The project entitled “E Nursery Shop” is developed using HTML, CSS and javascript as frontend and Python programming language and Sqlite database in backend to computerize the process of online shopping. This project covers only the basic features required.
    }
        \section{Features of “E Nursery”}
        \paragraph
        {
        “E Nursery System” provides various features, which complement the information system and increase the productivity of the system. These features make the system easily usable and convenient.  Some of the important features included are listed as follows:
        }
        \begin{itemize}
            \item Intelligent User Forms Design
            \begin{itemize}
                \item Data access and manipulation through same forms
                \item Access to most required information
            \end{itemize}
            
            \item Data Security 
            \item Restrictive data access, as per login assigned only.
            \item Organized and structured storage of facts.
            \item Strategic Planning made easy.
            \item No decay of old Records.
            \item Exact financial position of the Business.
        \end{itemize}
        
        \section{Benefits Accrued from “E Nursery”}
        \begin{itemize}
            \item Intelligent User Forms Design
            \begin{itemize}
                \item Data access and manipulation through same forms
                \item Access to most required information
            \end{itemize}
            
            \item Data Security 
            \item Restrictive data access, as per login assigned only.
            \item Organized and structured storage of facts.
            \item Strategic Planning made easy.
            \item No decay of old Records.
            \item Exact financial position of the Business.
        \end{itemize}
        \section{Limitations of  “E Nursery”}
        \paragraph
        {
        Besides the above achievements and the successful completion of the project, we still feel the project has some limitations, listed as below:
        }
        
        \begin{enumerate}
            \item It is not a large scale system. 
            \item Only limited information provided by this system. 
            \item Since it is an online project, customers need internet connection to buy products
            \item People who are not familiar with computers can’t use this software.
        \end{enumerate}
    \chapter{Future Scope}
    \paragraph
    {
    This web application involves almost all the features of the online shopping. The future implementation will be online help for the customers and chatting with website administrator.
    }
    
    \chapter{Bibliography}
    
    \begin{thebibliography}{20}

      \bibitem{wikipedia} wikipedia, https://en.wikipedia.org/wiki/Plant\_nursery
    
      \bibitem{Geeks for Geek} Geeks for Geek, https://www.geeksforgeeks.org/python-django/
    
      \bibitem{javatpoint} javatpoint, https://www.javatpoint.com
    
      \bibitem{pythonOfficial} Python Official, https://www.python.org/
    
    \bibitem{tutorialspoint}tutorialspoint,  https://www.tutorialspoint/
    \bibitem{REFERENCE BOOKS}  REFERENCE BOOKS.Two scoops of Django for 1.11 by Daniel Greenfeld’s and Audrey Greenfield
     \bibitem{REFERENCE BOOKS}  REFERENCE BOOKS.Lightweight Django by Elman and Mark Lavin
      \end{thebibliography}
    
\end{document}
